\documentclass[a4paper]{article}

\usepackage{listings}
\usepackage{xspace}
\usepackage[utf8]{inputenc}
\usepackage[frenchb]{babel}
\usepackage{graphicx}
\usepackage{xspace}
\usepackage{multirow}
\usepackage{rotating}
\usepackage{wrapfig}
\usepackage{hyperref}
\usepackage{breakurl}

\usepackage{wasysym}

\newcommand{\pco}{{\tt PathCrawler-online.com}}
\newcommand{\SANTE}{\ensuremath{\mbox{\scshape Sante}}\xspace}
\newcommand{\SANTEController}{\textsc{Sante\-Controller}\xspace}
\newcommand{\PathCrawler}{\textsc{Path\-Crawler}\xspace}
\newcommand{\Framac}{\ensuremath{\mbox{\scshape Frama-C}}\xspace}
\newcommand{\FramaC}{\ensuremath{\mbox{\scshape Frama-C}}\xspace}
\newcommand{\COLIBRI}{\ensuremath{\mbox{\scshape Colibri}}\xspace}
\newcommand{\ACSL}{\ensuremath{\mbox{\scshape ACSL}}\xspace}
\newcommand{\PDG}{\ensuremath{\mbox{\scshape Pdg}}\xspace}
\newcommand{\Value}{\ensuremath{\mbox{\scshape Value}}\xspace}
\newcommand{\Slicing}{\ensuremath{\mbox{\scshape Slicing}}\xspace}
\newcommand{\From}{\ensuremath{\mbox{\scshape From}}\xspace}
\newcommand{\Jessie}{\ensuremath{\mbox{\scshape Jessie}}\xspace}
\newcommand{\WP}{\ensuremath{\mbox{\scshape Wp}}\xspace}
\newcommand{\SMTLIB}{\ensuremath{\mbox{\scshape SMT-LIB}}\xspace}

\newcommand{\safe}{\textit{safe}}
\newcommand{\bug}{\textit{bug}}
\newcommand{\unknown}{\textit{unknown}}
\newcommand{\alarms}{\mathop{\textit{alarms}}\nolimits}

\newcommand{\todo}{\textbf{TODO}}
\usepackage{color}

\newcommand{\commentNK}[1]{\color{red}((NK: {#1}))\color{black}}
\newcommand{\commentPC}[1]{\color{orange}((PC: {#1}))\color{black}}
\newcommand{\commentBY}[1]{\color{blue}((BY: {#1}))\color{black}}
\newcommand{\commentBM}[1]{\color{purple}((BM: {#1}))\color{black}}
\newcommand{\commentOC}[1]{\color{green}((OC: {#1}))\color{black}}
\newcommand{\commentAP}[1]{\color{gray}((AP: {#1}))\color{black}}

\title{\COLIBRI}
\author{Bruno Marre\and Benjamin Blanc\and Patricia Mouy\and Zakaria Chihani\and
  Franck Vedrine\and François Bobot}

\begin{document}

\maketitle

\COLIBRI{}  is a library
(COnstraint LIBrary for ve\-RI\-fi\-cation) developed at
CEA LIST and used for verification or test data generation purposes
since 2000, using the techniques of constraint programming.
The variety of types and constraints provided by \COLIBRI{} makes it possible to
use it in many testing  tools at CEA LIST like GATeL \cite{gatelMBT10}, for model based
testing from Lustre/SCADE, and Osmose \cite{BardinH08},
for structural testing from binary code, \PathCrawler{} tool for
concolic C testing.

\section{General presentation}\label{SubsecColibriGeneral}

\COLIBRI{} is a finite domain solver that use usual constraint
programming techniques. The basis is that one or many domains are
attached to each terms, and the constraints tighten the domains of one
term using the domains of the other terms involve in the constraint.
When no constraints could improve anymore the domain of any term, a
decision is made. Usually it splits the domain in two, but other
techniques can be used as 3B filtering which decide on the extremes of
the domains and if unsatisfiable try to improve this bound.

The domains are very specific, no reduction to simpler theory like
bit-blasting is used. \COLIBRI uses a domain of union of intervals for
real and integer inter-reduced with a domain of congruence. For
floating points it uses a domain of interval. For bitvector a domain
that indicates if the bits are set, unset or unknown.

In addition to these domains which reason on local properties, \COLIBRI
uses for integer, reals, and floating point DBM. The DBM use lots of
patterns to be able to do limited but useful non-linear reasoning.

The power of the reasoning of \COLIBRI, in particular in the floating
point, are due to the information sharing and inter dependences of all these
reasoning techniques.

The combination of all the components of \COLIBRI is simplified since,
like all CP-solver, all the domains and constraints are improving the
same model.

% \COLIBRI{} %mainly
% provides basic constraints for arithmetic operations and
% comparisons of various numeric types (integers, reals and floats). Cast constraints
% are available for cast operations between these types. \COLIBRI{} also provides
% basic procedures to instantiate variables in their domains
% making it possible to
% design different instantiation
% strategies (or \textit{labeling procedures}). These
% implement specific heuristics
% to determine the way the variables should be instantiated during constraint resolution
% (e.g. a particular order of instantiation)
% and the choice of values inside their domain
% (e.g. trying boundary or middle values first).
% Thus the three aforementioned testing tools have
% designed their own labeling procedures  on the basis of \COLIBRI{} primitives.


% The domains of numerical variables are represented by %mapped to
% unions of disjoint intervals with
% finite bounds (no infinities or NaNs): integer bounds for integers, double float
% bounds for reals, and double/simple float bounds for double/simple floating point
% formats. These unions of intervals make it possible to accurately handle domain
% differences.
% For each numeric type and each basic unary/binary operation or
% comparison, \COLIBRI{} provides the corresponding constraint.

% Moreover, for each arithmetic operation, additional filtering rules apply %simple
% algebraic simplifications, which are very similar for integer and real arithmetics,
% whereas floating arithmetics uses specific rules. Here are a few examples
% of basic algebraic simplifications.
% \begin{itemize}
% \item \textit{Factorization of constraints.}
% If the constraint store contains
% $A+B=C$ and a new contraint $A+B=X$ is added (or derived)
% in the store, then
% only one of these two constraints remains in the store
% and the variables $C$ and $X$  are  \textit{unified},
% that is, they are identified and their domain is set to the intersection
% of the domains of $C$ and $X$.
% \item \textit{Special values (neutral elements, absorbing elements).}
% The constraint $A+0=X$ leads to the unification of $A$ and $X$.
% \item \textit{Identities between arguments.}
% The constraint $A+A=C$ is transformed into $2*A=C$ which is more
% precisely handled by interval arithmetics.
% The constraint $A+X=A$ in integer and real arithmetics
% leads to the unification of $X$ with $0$  (which is not valid
% in floating point arithmetics, cf Section~\ref{SubsecColibriRealFloat}).
% \end{itemize}

\section{Bounded and modular integer arithmetics}\label{SubsecColibriIntArithmetics}
\COLIBRI{} provides two kinds of arithmetics for integers: bounded arithmetics for ideal
finite integers and modular arithmetics for signed/unsigned computer integers.

Bounded arithmetics is implemented with classical filtering rules for integer interval arithmetics.
These rules are managed in the projection functions of each
arithmetic constraint.
Moreover, a congruence domain is associated to each integer variable.
Filtering rules handle these congruences in order to compute new
ones and maintain the consistency of interval bounds with
congruences (as in \cite{LeconteB06}).
The congruences are introduced by multiplications by a constant
and propagated in the projection functions of each arithmetic constraint.



Modular arithmetics constraints are implemented by a combination of
bounded arithmetics constraints with modulus constraints as detailed in
\cite{GotliebLM10}. Thus they benefit from the mechanisms provided for bounded integer
arithmetics. Notice that using the unions of disjoint intervals for the domain
representation makes it possible to precisely represent the domain of
signed/unsigned integers. For example, consider the constraint $A +_{2^3} B = C$ over 3-bit
unsigned integers where $A \in 2..4$, $B \in 4..7$ and $C \in 0..7$.
This constraint is handled by the constraints corresponding to the bounded
arithmetic expression $C = A + B - K * 8$
where $K \in 0..1$ represents the overflow status.
The filtering of these constraints converges to the interval $C \in [0..3, 6..7]$
where the sub-interval $0..3$ of $C$ is reached when there is an overflow (i.e. $K = 1$).
The domain representation by compact intervals in this case would be less precise and
result in a complete interval $C \in [0..7]$ without any reduction.

% \section{Example of test generation in bounded vs. modular arithmetics}\label{SubsecColibriExample}
% %\paragraph{Example of test generation in bounded vs. modular arithmetics.}

% Suppose that the constraints for the program
% path $\pi$ executing the lines 3,4,5,6  in the program below
% are incrementally posted to \COLIBRI{} in order to generate a test for this path.
% \begin{lstlisting}[stepnumber=1]
% int f(int x) {
%   int y;
%   if(x >= 0)
%     y = x + 1;
%   if(y < x)
%     y = 0;
%   return y;
% }
% \end{lstlisting}
% Assume that logical variables $X$ and $Y$ represent the values of \verb'x' and \verb'y', and
% the initial domain of the input is $X\in [ MinInt..MaxInt ]$.
% Posting the constraint $X\ge 0$ reduces the domain of $X$ to  $[ 0..MaxInt ]$
% in any kind of arithmetics.
% If bounded arithmetics (without overflows) is used, then
% posting the second constraint $Y=X+1$ activates filtering rules for the $plus$ constraint
% and results in the following constraints and domains:
% $$Y=X+1, \ \ X\in [ 0 \enskip .. \enskip MaxInt-1 ],  \ \ Y\in [ 1 \enskip .. \enskip MaxInt ].$$
% Posting the third constraint $Y < X$ activates filtering rules again, and \COLIBRI{}
% reports that the constraints have no solution. The desired path $\pi$ cannot
% be activated without overflows.

% If modular arithmetics (with authorized overflows) is chosen,
% then $MaxInt+1=MinInt$, thus
% posting the second constraint $Y=X+1$ results in
% $$Y=X+1, \ \ X\in [0\enskip ..\enskip MaxInt],  \ \ Y\in [ MinInt ]\cup[1\enskip ..\enskip MaxInt ].$$
% When the third constraint $Y < X$ is posted, \COLIBRI{} finds the unique solution
% and generates a test $X=MaxInt$ activating the chosen path $\pi$.

\section{Real and floating point arithmetics}\label{SubsecColibriRealFloat}
Real arithmetics is implemented with classical
filtering rules for real interval arithmetics
where interval bounds are double floats. In real interval arithmetics
each projection function is computed using different rounding modes for the lower
and the upper bounds of the resulting intervals. The lower bound is computed by
rounding downward, towards $-1.0\mathit{Inf}$ (i.e. $-\infty$),
while the upper bound is computed by rounding
upward, towards $+1.0\mathit{Inf}$ (i.e. $+\infty$).
This enlarging ensures that the resulting interval
is the smallest interval of doubles including all real solutions.

Floating point arithmetics is implemented with a specific interval arithmetics as
introduced by Claude Michel in \cite{Michel02}. Notice that properties like
associativity or distributivity do not hold in floating point calculus.
The projection functions in this arithmetics have to take into account
{\em absorption} and {\em cancellation} phenomena specific to floating point
computations. These phenomena are handled by specific filtering rules allowing to
further reduce  the domains of floating point variables. For example,
%the identity occurring in
the constraint $A +_F X = A$ over floating point numbers
%shows
means that $X$ is absorbed by $A$.
The minimal absolute value in the domain of $X$ can be used to eliminate all the
values in the domain of $A$ that do not absorb this minimum. Thus, in double
precision with the default rounding mode (called {\em nearest to even}), for
$X = 1.0$ the domain of $A$ is strongly reduced to the union of two interval
of values that can absorb $X$:
$$[\mathit{MinDouble}\, ..\, -9007199254740996.0, \ 9007199254740992.0\, ..\, \mathit{MaxDouble}].$$

\COLIBRI{} uses very general and powerful filtering rules for addition
and subtraction operations as described in \cite{MarreM10}. For example, for the
constraint $A+B=1.0$ in double precision with the {\em nearest to even} rounding mode,
such filtering rules converge to the same interval for $A$ and $B$
$$[-9007199254740991.0\ ..\ 9007199254740992.0].$$

\section{Bitvectors}

\COLIBRI{} doesn't use bit-blasting but a domain specific to
bitvectors\cite{Michel2012}. It has been recently reimplemented. This
domain is implemented with two integers (GMP), one that specify the
bits that are known to not be set, the other that specify the bits
that are known to be set. So all the propagations for the bitwise
operations could be efficiently implemented. The bitvector terms are
also seen as bounded integer terms. So the bitvector terms have two
domains, bitvector and bounded terms with modulo, that are
inter-reduced. The propagators for the arithmetic and bitwise
bitvector operations work on both domains like in \cite{BardinHP10},
where benchmarks show that \COLIBRI{} can be competitive with powerful
SMT solvers of 2010.

\section{Implementation details}\label{SubsecColibriImplementation}

\COLIBRI{} is implemented in ECLiPSe Prolog \cite{Eclipse11}. Its suspensions,
generic unification and meta-term mechanisms make it possible to easily
design new abstract domains and associated constraints.
Incremental constraint posting with on-the-fly filtering
and automatic backtracking to a previous constraint state provided by \COLIBRI{}
are important benefits for
search-based state exploration tools, and in particular, for
test generation tools.

To conclude this presentation of \COLIBRI{}, let
us remark that the accuracy of its implementation relies a lot on the use of union
of intervals and the combination of abstract domain filtering rules with algebraic
simplifications.
% with big integer bounds or double float bounds. This could be
%considered very expensive for constraints systems involving heavy computations.
%However,
% Experiments in \cite{BardinHP10} using \SMTLIB
% benchmarks show that \COLIBRI{} can be competitive with powerful SMT solvers.


\section{SMTCOMP 2016}

Since the beginning \COLIBRI have been used at CEA, as an internal
library. During the last months, as part of the SOPRANO project
(ANR-14-CE28-0020), an effort have been made to make a standalone
program from this library that could read and solve \SMTLIB problems.
We focused on \texttt{QF\_FP} and \texttt{QF\_BV} because the technique
used for solving these problems in the CP community (specific domains,
propagation) are very different from the one used in SMT community
(bit-blasting).

However the semantic of \COLIBRI have been guided by its usage:
\begin{itemize}
\item The tools that use \COLIBRI always check the absence of RTE
  (runtime errors eg divide by zero). So \COLIBRI
  could make the hypothesis that all RTE are impossible and reduce the
  domains accordingly.
\item NaN and infinite are considered as RTE by the tools.
\item On floating points, the IEEE equality is the only equality used.
\item The API on floating points is limited to floating points
  operators so $+0$ and $-0$ can be confounded (division by zero is an
  RTE).
\item The rounding mode \emph{Nearest to even} is the only one
  implemented, the others are never used except in very specialized
  programs where they are used scarcely due to the cost of changing
  the rounding mode.
\end{itemize}

More generally \COLIBRI threat partial functions using a three valued
logic, where a problem is unsatisfiable if it is interpreted as false
or unknown in every model. \SMTLIB threat partial functions by making
them total and letting models use an arbitrary value for the not
defined cases. So \COLIBRI considers unsatisfiable problems that the
\SMTLIB considers satisfiable.

In order to bridge the gap between this two semantics we used an RTE
checker written in OCaml and that use Z3 as a backend. We use the
following step for solving the \SMTLIB benchmarks:

\begin{enumerate}
\item Tidy the benchmark (remove set-info, rename variables with complicated
  names eg. \verb&|c::main::1::a!0@1#1|&,...)
\item Send the benchmark to \COLIBRI
  \begin{enumerate}
  \item If \COLIBRI answer SAT, check that the models is still one
    when $+0$ and $-0$ are distinguished.
    \begin{enumerate}
    \item If it is the case, return SAT
    \item Otherwise, return UNKNOWN
    \end{enumerate}
  \item If \COLIBRI answer UNKNOWN, return UNKNOWN
  \item If \COLIBRI answer UNSAT:
    \begin{enumerate}
    \item Generate a formula that checks if it exists a model that
      interprets the formula as undefined in \COLIBRI semantic (eg,
      the result of an application or variable is NaN) and that
      satisfy the formula in \SMTLIB semantic. Send it
      to Z3.
      \begin{enumerate}
      \item If Z3 answer SAT, return SAT
      \item If Z3 answer UNKNOWN, return UNKNOWN
      \item If Z3 answer UNSAT, return UNSAT
      \end{enumerate}
    \end{enumerate}
  \end{enumerate}
\end{enumerate}


\bibliographystyle{abbrv}
\bibliography{biblio2}

\end{document}



% Local variables:
% mode: latex
% TeX-master: "main.tex"
% End:
